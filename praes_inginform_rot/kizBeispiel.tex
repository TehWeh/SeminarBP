\documentclass{beamer}
\mode<presentation>
{
  \usetheme{myulm}
  \setbeamercovered{transparent}
  \setbeamertemplate{navigation symbols}{} % no navigation bar
  \setbeamersize{sidebar width left=1.17cm}
}

\usepackage[ngerman]{babel}
\usepackage[utf8]{inputenc}
\usepackage{amsmath,amssymb,amsfonts}
\usepackage{times}
\usepackage{graphicx}
\usepackage{fancyvrb}
\usepackage{array}
\usepackage{colortbl} % ING INF PSY
\usepackage{tikz}
\usepackage{listings}
\usepackage{verbatim}
\usetikzlibrary{arrows, decorations.markings}

% Anfang der Titelfolie
% Anpassung von: Titel, Untertitel, Autor, Datum und Institut

\title{Belief Propagation}
%\subtitle{Manchmal wird ein Untertitel ben\"{o}tigt, der ebenfalls einzeilig oder mehrzeilig sein kann}
\author{Michael Ruderer}
\newcommand{\presdatum}{\today} % alternativ zu \today: Eingabe eines festen Datums
\institute{Uni Ulm\\}
%Ende der Titelfolie


% Anfang der Kopfzeile der Folien
% Anpassung von: Zwischentitel, Leitthema oder Name
% Das Datum wird oben geändert: unter \presdatum{}!

\newcommand{\zwischentitel}{\insertsection}
\newcommand{\leitthema}{Belief Propagation}
% Ende der Kopfzeile

% Anfang der Folien
\begin{document}
\hspace*{-1.49cm}
\frame[plain]{\titlepage}

% Das Inhaltsverzeichnis
\hspace*{-0.7cm}
\section*{Inhalt/Überblick} % diese Section erscheint nicht im Inhaltsverzeichnis
\begin{frame}
  \frametitle{Inhaltsverzeichnis}
  \tableofcontents
\end{frame}


% 1. Folie
\section{Begriffe}

\begin{frame}
	\frametitle{SAT}
\vspace{-0.5cm}
  \begin{itemize}
    \item SAT formula in CNF
    $$\mathcal{F} = (x_1 \lor x_2 \lor \overline{x_3}) \land (x_3 \lor x_4)$$
    \vspace{-0.5cm}
    	\begin{itemize}
    		\item Boolean variables $x_1, x_2, \ldots, x_n$
    		\item Negations $\overline{x_1}, \ldots, \overline{x_n}$
    		\item Clauses: Disjunction of variables and their negations
    		\item $\mathcal{F}$: Conjuction of clauses
    	\end{itemize}
    %\item local functions $\cong$ clauses
    \item Is there an assignmnent of the variables that satisfies $\mathcal{F}$?
    \item How does the assignment look like?
  \end{itemize}
\end{frame}

\begin{frame}
	\frametitle{Factor Graphs}
	\vspace{-0.5cm}
		Factor graphs represent a function's factorization
		
		\begin{itemize}
			\item Function $f(X)$ over variables $X = \{x_1, x_x, \ldots, x_n\}$
			\item Global function $f$ factorizes to local functions
			$$ f(X) = \prod_{j=1}^m f_j(S_j)$$
			\item Local functions have smaller input $S_j \subset X$
		\end{itemize}
\end{frame}

\begin{frame}
	\frametitle{Factor Graphs}
		Factor graphs represent a function's factorization
		
		\begin{itemize}
			\item Two types of nodes
				\begin{itemize}
					\item Variable nodes: represent variables
					\item Factor nodes  : represent local functions
				\end{itemize}
			\item Edges connect variable and factor nodes
			\item Factor nodes are connected to all variable nodes of their input variables
		\end{itemize}
\end{frame}


\begin{frame}
	\frametitle{Example}
		\begin{align*}
		f(x_1, x_2, x_3) &= x_1^3 - x_1^2x_2 + x_1^2x_3 - x_1x_2x_3 \\
		&= \underbrace{(x_1)}_{f_1(x_1)} * \underbrace{(x_1 - x_2)}_{f_2(x_1, x_2)} * \underbrace{(x_1 + x_3)}_{f_3(x_1, x_2)}
\end{align*}		 

\begin{figure}
\centering

\begin{tikzpicture}[scale=0.7,transform shape]
   	\node[shape=circle,draw=black] (x1) at (0,0) {$x_1$};
    \node[shape=circle,draw=black] (x2) at (-2,-2) {$x_2$};
    \node[shape=circle,draw=black] (x3) at (2,-2) {$x_3$};
    \node[rectangle,draw=black, label = {$f_1$}, fill] (f1) at (0,2) {};
    \node[rectangle, fill, draw=black, label = {$f_2$}] (f2) at (-1,-1) {};
    \node[rectangle, fill, draw=black, label = {$f_3$}] (f3) at (1, -1) {} ;

    \path [-] (x1) edge node[left] {} (f1);
    \path [-] (x1) edge node[left] {} (f2);
    \path [-] (x1) edge node[left] {} (f3);
    \path [-] (f2) edge node[left] {} (x2);
    \path [-] (f3) edge node[left] {} (x3);
   
\end{tikzpicture}


\end{figure}
\end{frame}



\begin{frame}
	\frametitle{Factor Graphs}
	Factor Graph of a CNF formula
\begin{minipage}[0.2\textwidth]{\textwidth}
\begin{columns}[T]
\begin{column}{0.5\textwidth}
		
		\begin{align*}
			\mathcal{F} \;= \;&(x_1 \lor x_2 \lor {x_3}) \land (\overline{x_1} \lor x_2 \lor x_4) \\ & \land (\overline{x_3} \lor x_4) \land (\overline{x_1} \lor \overline{x_2})
		\end{align*}
		\begin{itemize}
			\item $\mathcal{F}$ is a product of clauses
			\item Clauses $\cong$ local functions
		\end{itemize}
	\end{column}
	\begin{column}{0.5\textwidth}
\begin{figure}
\centering

\begin{tikzpicture}[scale=0.8,transform shape]
   	\node[shape=circle,draw=black] (x1) at (3,0) {$x_1$};
    \node[shape=circle,draw=black] (x2) at (1,-2) {$x_2$};
    \node[shape=circle,draw=black] (x3) at (-1,-2) {$x_3$};
    \node[shape=circle,draw=black] (x4) at (1,-4) {$x_4$};
    
    \node[rectangle,draw=black, label = {$a$}, fill] (a) at (0,0) {};
     \node[rectangle,draw=black, label = {$d$}, fill] (d) at (2,-1) {};
     \node[rectangle,draw=black, label = {$b$}, fill] (b) at (4,-2) {};
     \node[rectangle,draw=black, label = {$c$}, fill] (c) at (0,-3) {};
     
 
    \path [-] (x1) edge node[left] {} (a);
    \path [-] (x2) edge node[left] {} (a);
    \path [-] (x3) edge node[left] {} (a);
    
    \path [dashed] (x1) edge node[left] {} (d);
    \path [dashed] (x2) edge node[left] {} (d);
    
    \path [-] (x2) edge node[left] {} (b);
    \path [dashed] (x1) edge node[left] {} (b);
    \path [-] (x4) edge node[left] {} (b);
    
    \path [-] (x4) edge node[left] {} (c);
    \path [dashed] (x3) edge node[left] {} (c);
\end{tikzpicture}
\end{figure}
\end{column}
\end{columns}
\end{minipage}	
\end{frame}


\begin{frame}
	\frametitle{Message Passing}
	Message Passing Algorithms on factor graphs
	\begin{itemize}
		\item Nodes communicate through messages
		\item Messages are passed over the graph's edges
		\item Two types of messages
			\begin{itemize}
				\item $\mu_{i \rightarrow a}$ sent from factor $a$ to variable $i$
				\item $\mu_{a \rightarrow i}$ sent from variable $i$ to factor $a$
			\end{itemize}	
	\end{itemize}
	\begin{figure}
\centering

\begin{tikzpicture}[scale=0.8,transform shape, >=stealth]
   	\node[shape=circle,draw=black] (i) at (3,0) {$i$};
   \tikzset{myptr/.style={decoration={markings,mark=at position 1 with %
    {\arrow[scale=3,>=stealth]{>}}},postaction={decorate}}}
    
    \node[rectangle,draw=black, label = {$a$}, fill] (a) at (0,0) {};
    
 	\only<1>{    \path [-] (i) edge node[left] {} (a);}

    \only<2> {\path [->] (i) edge node[left, above] {$\mu_{i \rightarrow a}$} (a);}
    \only<3>{\path [<-] (i) edge node[left, above] {$\mu_{a \rightarrow i}$} (a);}
    
    %\path [dashed] (x3) edge node[left] {} (c);
\end{tikzpicture}
\end{figure}
		
\end{frame}

\begin{frame}
	\frametitle{Message Passing}
	Message Passing Algorithms on factor graphs
	\begin{itemize}
		\item Message $\mu_{a \rightarrow i}$ determined by incoming messages $\mu_{j \rightarrow a}$ from neighbours $j \neq i$
		\item Computation Rule depends on application
			
	\end{itemize}
\begin{figure}
\centering

\begin{tikzpicture}[scale=0.8,transform shape, >=stealth]
   	\node[shape=circle,draw=black] (i) at (3,0) {$i$};
   \tikzset{myptr/.style={decoration={markings,mark=at position 1 with %
    {\arrow[scale=3,>=stealth]{>}}},postaction={decorate}}}
    
    \node[rectangle,draw=black, label = {$a$}, fill] (a) at (0,0) {};
    
	\node[shape=circle,draw=black] (j1) at (-1.8,1.2) {};    
	\node[shape=circle,draw=black] (j2) at (-2,0) {};    
	\node[shape=circle,draw=black] (j3) at (-1.8,-1.2) {};    

    
 	\path [->] (a) edge node[left, above] {$\mu_{a \rightarrow i}$} (i);
 	
 	\path [->] (j1) edge node[left, above] {$\mu_{j_1 \rightarrow a}$} (a);
 	\path [->] (j2) edge node[left, above] {$\mu_{j_2 \rightarrow a}$} (a);
 	\path [->] (j3) edge node[right, below] {$\mu_{j_3 \rightarrow a}$} (a);

   
    
    %\path [dashed] (x3) edge node[left] {} (c);
\end{tikzpicture}
\end{figure}
		
\end{frame}



\begin{frame}
	\frametitle{Message Passing}
	Message Passing Algorithms on trees
	\begin{itemize}
		\item Messages genereated bottom up
		\item Leaves start sending messages
		\item Messages are propagated forward in the tree
		\item Does \textbf{not} work on graphs with cycles
	\end{itemize}
\end{frame}

\begin{frame}
\frametitle{Example}
\begin{figure}[h]
\centering

\begin{tikzpicture}[scale=0.8,transform shape]
   	\node[rectangle,draw=black, label = {$a$}, fill] (i) at (3,0) {$a$};
    \node[shape=circle,draw=black] (a) at (5,0) {$i$};
   
  \node[shape=circle,draw=black] (b1) at (0,2) {$j_1$};
  \node[shape=circle,draw=black] (b2) at (0,0) {$j_2$};
  \node[shape=circle,draw=black] (b3) at (0,-2) {$j_3$};

	\node[rectangle,draw=black, label = {$b_1$}, fill] (j1) at (-3,3) {};
	\node[rectangle,draw=black, label = {$b_2$}, fill] (j2) at (-3,1) {};
	\node[rectangle,draw=black, label = {$b_3$}, fill] (j3) at (-3,-1) {};
	\node[rectangle,draw=black, label = {$b_4$}, fill] (j4) at (-3,-3) {};


    \only<1-1>{\draw[-, >= stealth] (b1) edge [right] node {} (j1);}
	\only<2-13>{\draw[<-, >= stealth] (b1) edge [right] node {} (j1);}
	\only<14-17>{\draw[<->, >= stealth] (b1) edge [right] node {} (j1);}
	
	\only<1-2>{\draw[-, >= stealth] (b2) edge [right] node {} (j2);}
	\only<3-14>{\draw[<-, >= stealth] (b2) edge [right] node {} (j2);}
	\only<15-17>{\draw[<->, >= stealth] (b2) edge [right] node {} (j2);}
	
	\only<1-3>{\draw[-, >= stealth] (b2) edge [right] node {} (j3);}
	\only<4-15>{\draw[<-, >= stealth] (b2) edge [right] node {} (j3);}
	\only<16-17>{\draw[<->, >= stealth] (b2) edge [right] node {} (j3);}
	
	\only<1-4>{\draw[-, >= stealth] (b3) edge [right] node {} (j4);}
	\only<5-16>{\draw[<-, >= stealth] (b3) edge [right] node {} (j4);}
	\only<17-17>{\draw[<->, >= stealth] (b3) edge [right] node {} (j4);}
	
	\only<1-5>{\draw[-, >= stealth] (i) edge [right] node {} (a);}
	\only<6-11>{\draw[<-, >= stealth] (i) edge [right] node {} (a);}
	\only<12-17>{\draw[<->, >= stealth] (i) edge [right] node {} (a);}
	
	\only<1-6>{\draw[-, >= stealth] (b1) edge [right] node {} (i);}
	\only<7-12>{\draw[->, >= stealth] (b1) edge [right] node {} (i);}
	\only<13-17>{\draw[<->, >= stealth] (b1) edge [right] node {} (i);}
	
	\only<1-7>{\draw[-, >= stealth] (b2) edge [right] node {} (i);}
	\only<8-9>{\draw[->, >= stealth] (b2) edge [right] node {} (i);}
	\only<10-17>{\draw[<->, >= stealth] (b2) edge [right] node {} (i);}
	
	\only<1-8>{\draw[-, >= stealth] (b3) edge [right] node {} (i);}
	\only<9-10>{\draw[->, >= stealth] (b3) edge [right] node {} (i);}
	\only<11-17>{\draw[<->, >= stealth] (b3) edge [right] node {} (i);}
	
\end{tikzpicture}
%\caption{Subgraph of a factor tree with all messages required for computing $m_{a \rightarrow i}$}
\end{figure}
\end{frame}

\begin{frame}
	\frametitle{Message Passing}
	\begin{itemize}
		\item In general graphs: \emph{Loopy} Message Passing
		\item Randomly initialize all messages
		\item Apply the Update rule until messages have converged
		\item Scheduling important
	\end{itemize}

		
\end{frame}

\begin{frame}[containsverbatim]
	\frametitle{Message Passing}
	Generic Message Passing Algorithm
	\begin{lstlisting}[mathescape = true, gobble=15, basicstyle=\ttfamily]
		1. Randomly initialize all warnings $\mu_{i \rightarrow a}, \mu_{a \rightarrow i}$
		2. For $t = 0$ to $t_{max}$
		  2.1 Apply the update rule to all edges
		      in random order
		  2.2 If no message has changed goto 3
		3. If $t = t_{max}$ return UNGONVERGED
		   Else      return the converged messages
	\end{lstlisting}
\end{frame}

\begin{frame}
	\frametitle{Message Passing}
	\begin{itemize}
		\item Loopy Message Passing converges on trees
		\item On cyclic graphs
			\begin{itemize}
				\item no guarantee of convergence
				\item no guarantee of correctness
			\end{itemize}
		\item Can be used as heuristics
		\item In practice often correct
			\end{itemize}

		
\end{frame}


\section{Warning Propagation}
\section{Warning Propagation}

The messages used in the Warning Propagation Algorithm (WP) presented in [\cite{dummyzitat}] are called \emph{warnings}. A warning $u_{a\rightarrow i} \in \{0, 1\}$ is passed from clause $a$ to variable $i$. A converged warning  $u^{\star}_{a\rightarrow i}$ with value $1$ should indicate, that  to satisfy the clause $a$, the variable $i$ has to take the value $1$ if $j \in V_+(a)$ or $0$ if $j \in V_-(a)$. The warning $u^{\star}_{a \rightarrow i}$ will \emph{fix} the variable $i$. 
\subsection{Propagation Algorithm}

Like the general algorithm described in section \ref{BPA} warning propagation is correct on trees and can be used as a heuristic for cyclic graphs by randomly initializing the warnings and hoping for convergence.

The algorithm starts by assigning each warning $u_{a\rightarrow i}$ a random starting value and updates these provisional warnings until their values have converged to a set of fixed point warnings $u^{\star}_{a \rightarrow i}$ or until the number of iterations has exceeded some limit $t_{max}$.

The general idea is that the clause $a$ has to fix the variable $i$ only if the all of its other variables $j \in V(a)\setminus i$ are already fixed to values that do not satisfy the clause $a$. \newline
The first step in the update procedure is to compute for each $j \in V(a) \setminus i$ the so called \emph{cavity field} $h_{j \rightarrow a}$ that indicates what value $j$ should take in the subproblem defined by $\tau_{j \rightarrow a}$. To compute $h_{j \rightarrow a}$ one has to count how many of the clauses $b \neq a$ fix $j$ to $1$ and how many fix $j$ to $0$:
$$h_{j \rightarrow a} = \sum_{b \in V_+(j)\setminus a}{u_{b \rightarrow j}} - \sum_{b \in V_-(j)\setminus a}{u_{b \rightarrow j}}$$

The clauses $b \in V_+(j)$ are the ones that would fix $i$ to $1$ if their warnings are active, the clauses $b \in V_-(j)$ would fix $i$ to $0$.
So if $h_{j \rightarrow a}$ is positive, the variable $i$ tends to the value $1$, if the cavity field is negative it tends to $0$. If $h_{j \rightarrow a} = 0$ which includes the case $V(j) \setminus a = \emptyset$ no conclusion can be made.

When all cavity fields are computed, each variable $j$ with $h_{j \rightarrow a} \neq 0$ has a preferred value. This preferred value either makes the clause $a$ satisfied or does not contribute to the clause. If all variables $j \in V(a)\setminus i$ prefer a non satisfying value, the clause $a$ sends a warning to $i$, meaning that $i$ should take the satisfying value.

This warning can be computed by $$u_{a \rightarrow i} = \prod_{j\in V(a)\setminus i}{\theta(h_{j \rightarrow a}J_j^a)} \quad \text{ with } \theta(x) = \left\{
  \begin{array}{@{}ll@{}}
    0, & \text{if}\ x \leq 0 \\
    1, & \text{otherwise}
  \end{array}\right\}$$

The factor $\theta(h_{j \rightarrow a}J_j^a)$ is $1$ if $j$ prefers to violate $a$ and $0$ if not:
\begin{itemize}
\item[] If $j$ has no prefered value its cavity field is $0$ and $\theta(h_{j \rightarrow a}J_j^a) = 0$ meaning no warning will be sent.
\item[] If the preferred value of $j$ satisfies $a$, $J_j^a$ and $h_{j \rightarrow a}$ have different signs and  $\theta(h_{j \rightarrow a}J_j^a)$ is again $0$.
\item[] If the preferred value of $j$ \emph{violates} $a$, $J_j^a$ and $h_{j \rightarrow a}$ have the same sign and $\theta(h_{j \rightarrow a}J_j^a) = 1$. If this is the case for all $j \neq i$ the product evaluates to $1$ and $a$ sends a warning to $i$.
\end{itemize}

\begin{lstlisting}[mathescape=true]
	Warning Propagation Algorithm
	
	0. Randomly initialize all warnings $u_{a \rightarrow i} \in_R \{0, 1\}$
	
	1. For $t=0$ to $t = t_{max}$
		1.1 Compute in random order for all edges (a, i) $u_{a \rightarrow i} := \prod_{j \in V(a) \setminus i} {\theta \left( -J_j^a \left( \sum_{b\in V(j) \setminus a}{J_j^b u_{b \rightarrow j}} \right) \right)}$
		1.2 If no message has changed goto 2.
	2. If $t = t_{max}$ return UN-CONVERGED, else return the generated warnings $u_{a \rightarrow i}^{\star}$
\end{lstlisting}

If the algorithm successfully returns a set of converged warnings, each warning $u^{\star}_{a \rightarrow i} = 1$ fixes the variable $i$ to the value satisfying a:



\subsection{Decimation Algorithm}

\begin{itemize}
\item Contradiction Numbers
\item Lemma: $u^{\star}_{a \rightarrow i} = 1 \Rightarrow \text{ i has a fixed value}$
\item Decimation Algorithm
\item Example on graph from last chapter
	  (Je nachdem wie viel Platz in Tabellenform oder grafisch)


\end{itemize} 
%\begin{lemma}
%If $u^{\star}_{a \rightarrow i} = 1$ the subproblem $a$ can not be satisfied if $i$ is removed from the set of its variables

%\begin{proof} 
%\begin{itemize}
%\item[] If the edge $(a, i)$ has level $1$, the clause $a$ only contains the variable $i$. Without $i$, it is an empty disjunction and evaluates to $0$
%\item[] If the edge $(a, i)$ has level $2$ the remaining variables $j \in V(a) \setminus i \neq \emptyset$ only appear in $a$. Their cavity fields must be $0$ so in any step of the algorithm $u^{\star}_{a \rightarrow i}$ is set to $0$.
%\item[] If the edge's level is r > 2 and $u^{\star}_{a \rightarrow i} = 1$ each $j\in V(a) \setminus i$ receives at least one warning from 
%\end{itemize}
%\end{proof}
%\end{lemma}

\newpage

\section{Belief Propagation}
\section{Belief Propagation} \label{BP}

Warning Propagation braucht bisher 2 Seiten, in der fertigen Version also etwa 4-5. Belief Propagation sollte in etwa den selben Umfang haben, etwas weniger Beschreibung nötig weil vieles gleich ist, dafür aber komplizierter.

\begin{itemize}

\item[Algorithmus] In \ref{BPA} Algorithmus beschreiben. Grundsätzliche  Vorgehensweise ist dieselbe wie WP, also "nur" unterschiedliche Arten von Nachrichten und neue Update-Regel beschreiben

\item[Lösungen] Hier zeigen, wie  die Anzahl der erfüllenden Belegungen und die Wahrscheinlichkeit für $x_i = 1$ aus den konvergierten Nachrichten berechnet wird. Wenn noch genug Seiten frei mit ausführlichem Beispiel

\end{itemize}

\subsection{Propagation Algorithm} \label{BPA}


\subsection{Marginal Propabilities}

\subsection{Number of satisfying assignments}
\section{Conclusion}
\frame{
	\frametitle{Conclusion}
}


\end{document}