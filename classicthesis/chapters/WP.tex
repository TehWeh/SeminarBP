\section{Warning Propagation}

The messages used in the \emph{Warning Propagation} Algorithm (WP) presented in \cite{survprob} are called \emph{warnings}. A warning $u_{a\rightarrow i} \in \{0, 1\}$ is passed from clause $a$ to variable $i$. A converged warning  $u^{\star}_{a\rightarrow i}$ with value $1$ should indicate, that  to satisfy the clause $a$, the variable $i$ has to take the value $1$ if $j \in V_+(a)$ or $0$ if $j \in V_-(a)$. The warning $u^{\star}_{a \rightarrow i}$ will \emph{fix} the variable $i$. 
\subsection{Propagation Algorithm}

Like the general algorithm described in section \ref{BPA} warning propagation is correct on trees and can be used as a heuristic for cyclic graphs by randomly initializing the warnings and hoping for convergence.

The algorithm starts by assigning each warning $u_{a\rightarrow i}$ a random starting value and updates these provisional warnings until their values have converged to a set of fixed point warnings $u^{\star}_{a \rightarrow i}$ or until the number of iterations has exceeded some limit $t_{max}$.

The general idea is that the clause $a$ has to fix the variable $i$ only if the all of its other variables $j \in V(a)\setminus i$ are already fixed to values that do not satisfy the clause $a$. \newline
The first step in the update procedure is to compute for each $j \in V(a) \setminus i$ the so called \emph{cavity field} $h_{j \rightarrow a}$ that indicates what value $j$ should take in the subproblem defined by $\tau_{j \rightarrow a}$. To compute $h_{j \rightarrow a}$ one has to count how many of the clauses $b \neq a$ fix $j$ to $1$ and how many fix $j$ to $0$:
$$h_{j \rightarrow a} = \sum_{b \in V_+(j)\setminus a}{u_{b \rightarrow j}} - \sum_{b \in V_-(j)\setminus a}{u_{b \rightarrow j}} = -\sum_{b \in V(j)} J^b_j u_{b \rightarrow j}$$

The clauses $b \in V_+(j)$ are the ones that would fix $i$ to $1$ if their warnings are active, the clauses $b \in V_-(j)$ would fix $i$ to $0$.
So if $h_{j \rightarrow a}$ is positive, the variable $i$ tends to the value $1$, if the cavity field is negative it tends to $0$. If $h_{j \rightarrow a} = 0$ which includes the case $V(j) \setminus a = \emptyset$ no conclusion can be made.

When all cavity fields are computed, each variable $j$ with $h_{j \rightarrow a} \neq 0$ has a preferred value. This preferred value either makes the clause $a$ satisfied or does not contribute to the clause. If all variables $j \in V(a)\setminus i$ prefer a non satisfying value, the clause $a$ sends a warning to $i$, meaning that $i$ should take the satisfying value.

This warning can be computed by $$u_{a \rightarrow i} = \prod_{j\in V(a)\setminus i}{\theta(h_{j \rightarrow a}J_j^a)} \quad \text{ with } \theta(x) = \left\{
  \begin{array}{@{}ll@{}}
    0, & \text{if}\ x \leq 0 \\
    1, & \text{otherwise}
  \end{array}\right\}$$

The factor $\theta(h_{j \rightarrow a}J_j^a)$ is $1$ if $j$ prefers to violate $a$ and $0$ if not:
\begin{itemize}
\item[] If $j$ has no preferred value its cavity field is $0$ and $\theta(h_{j \rightarrow a}J_j^a) = 0$ meaning no warning will be sent.
\item[] If the preferred value of $j$ satisfies $a$, $J_j^a$ and $h_{j \rightarrow a}$ have different signs and  $\theta(h_{j \rightarrow a}J_j^a)$ is again $0$.
\item[] If the preferred value of $j$ \emph{violates} $a$, $J_j^a$ and $h_{j \rightarrow a}$ have the same sign and $\theta(h_{j \rightarrow a}J_j^a) = 1$. If this is the case for all $j \neq i$ the product evaluates to $1$ and $a$ sends a warning to $i$.
\end{itemize}
\begin{lstlisting}[mathescape=true]
	Warning Propagation Algorithm
	
	0. Randomly initialize all warnings $u_{a \rightarrow i} \in_R \{0, 1\}$
	
	1. For $t=0$ to $t = t_{max}$
		1.1 Compute in random order for all edges (a, i) $u_{a \rightarrow i} := \prod_{j \in V(a) \setminus i} {\theta \left( -\sum_{b\in V(j) \setminus a}{(J_j^a J_j^b) u_{b \rightarrow j}}  \right)}$
		1.2 If no message has changed goto 2.
	2. If $t = t_{max}$ return UN-CONVERGED, else return the generated warnings $u_{a \rightarrow i}^{\star}$
\end{lstlisting}
% $u_{a \rightarrow i} := \prod_{j \in V(a) \setminus i} {\theta \left( -J_j^a \left( \sum_{b\in V(j) \setminus a}{J_j^b u_{b \rightarrow j}} \right) \right)}$

The following lemma shows that - on trees - the computed messages indeed serve the purpose described in the first paragraph: If the algorithm successfully returns a set of converged warnings, each warning $u^{\star}_{a \rightarrow i} = 1$ fixes the variable $i$ to the value satisfying a:

\begin{lemma}\cite{survprob} Let $a \rightarrow i$ be an edge on level $r$. \newline
If $u^{\ast}_{a \rightarrow i} = 1$ the SAT formula defined by $\tau_{a \rightarrow i}$ is not satisfiable.

\begin{proof} Induction on $r$.
\begin{itemize}
	\item If $r = 0$, $a$ is a leaf and $\tau_{a \rightarrow i}$ defines a formula that consists of one empty clause.
	\item If $r = 1$ the message $u^{\ast}_{a \rightarrow i}$ cannot be $1$: $a$ contains one or more variables $j \neq i$ which all are leaf nodes that pass the message $m_{j \rightarrow a} = 0$ to $a$. The update equation yields $m_{a \rightarrow i} = \prod_j \theta(0) = 0$.
	\item Let $r \geq 2$ and $u^{\ast}_{a \rightarrow i} = -1$. For each variable $j \neq i$ of $a$ there is at least one $b$ with $J_j^b J_j^a u^\star_{b \rightarrow j}= 1$ or equivalently $J_j^b \neq J_j^a$ and  $u^\star_{b \rightarrow j} = 1$. By induction the subproblem $\tau_{b \rightarrow j}$ is not satisfiable and any assignment in which $j$ supports $a$ must violate at least one clause on $\tau_{b \rightarrow j}$.
\end{itemize}
\end{proof}
\end{lemma}



\subsection{Decimation Algorithm}

The algorithm that computes satisfying assignments based on the WP results is called \emph{Warning Inspired Decimation}. CID uses WP as a subprocedure. For a satisfiable formula $\mathcal{F}$ after running WP a partial assignment is computed that can be applied to $\mathcal{F}$ to obtain a smaller formula. Running CID recursively on that smaller formula gives a complete satisfying assignments. CID detects after the first run of WP if $\mathcal{F}$ is not satisfiable.

To determine if a formula is satisfiable one has to compute certain values using the converged WP messages.

The \emph{local field} of a variable $i$ is defined as $H_i := - \sum_{b \in V(i)} J_i^b u^\ast_{b\rightarrow i}$. \newline $H_i$ is computed similar to $h_{i \rightarrow a}$ in the WP algorithm. While 
$h_{i \rightarrow a}$ gave the current tendency of $i$ when ignoring $a$, the local field of $i$ uses the converged messages and counts how often $i$ is fixed to $1$ or $0$. \newline
If a variable is both fixed to $1$ and $0$, the formula must be unsatisfiable. \newline
If no such contradiction appears a partial assignment can be obtained by setting each variable to the value it was fixed to: $1$ if $H_i > 0$ and $0$ if $H_i < 0$. If no variable is fixed (all $H_i$ are $0$), CID chooses a random variable and sets it to a random value.



%\begin{lemma}
%If $u^{\star}_{a \rightarrow i} = 1$ the subproblem $a$ can not be satisfied if $i$ is removed from the set of its variables

%\begin{proof} 
%\begin{itemize}
%\item[] If the edge $(a, i)$ has level $1$, the clause $a$ only contains the variable $i$. Without $i$, it is an empty disjunction and evaluates to $0$
%\item[] If the edge $(a, i)$ has level $2$ the remaining variables $j \in V(a) \setminus i \neq \emptyset$ only appear in $a$. Their cavity fields must be $0$ so in any step of the algorithm $u^{\star}_{a \rightarrow i}$ is set to $0$.
%\end{itemize}
%\end{proof}
%\end{lemma}

\newpage