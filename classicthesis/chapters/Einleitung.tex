\chapter{Introduction}

As many common problems naturally reduce to finding solutions of boolean formulas, SAT solvers are frequently used in practice and there is a need for finding fast and precise algorithms.  
The main part of this seminar thesis is Section \ref{BPFS} where two probabilistic methods for heuristically finding assignments of satisfiable SAT formulas are described. These methods are called \emph{Warning Propagation} and \emph{Belief Propagation}. Both are so called \emph{message passing} algorithms which are explained in section \ref{FGMP} together with basic terms and notations.

The belief propagation algorithm is a general method for probabilistic inference problems and was proposed by Judea Pearl in 1982. The generic algorithm is used in many different contexts like coding theory or image processing. It can be applied to SAT formulas by approximating marginal probabilities over the formula's solutions. These probabilities then yield valid variable assignments.

A belief propagation approach was used in the SAT solver "Dimetheus", which won gold medals in 2014 and 2016 in the international SAT competition (see \cite{satcomp}) for efficiently solving randomly generated satisfiable instances.

Section \ref{FGMP} starts with defining the factor graph representation of a SAT formula and describes the general procedure of message passing on factor graphs. \newline
In section \ref{BPFS} the two algorithms are presented and demonstrated on an example formula.